\documentclass[a4paper,12pt]{article}
\begin{document}
\title{Stability of Deep Neural Networks}
\author{Antoine Hebert, Ribal Aladeeb, Tristan Glatard, Yohan Chatelain}
\maketitle
\section{Introduction}

In the last decade, Deep Learning has proven to be one of the most
powerful techniques for solving machine learning problems both in academia
and industry. However Deep Learning is not immune to the reproducibility
problems encountered in the scientific community. For this reason we are
interested in evaluating how numerically stable neural networks are. We
mainly focus on two basic classification use cases of neural networks: MNIST
and CIFAR10. Our aim is to run Monte Carlo (MCA) Simulations during the inference
phase on networks trained for solving the aforementioned use cases. In this
way we approximate the number of signifigant digits in the output
predictions.

\section{Methods}

To perform MCA experiments we use Verificarlo, a tool for performing
automatic Monte Carlo Arithmetic on any C, C++, or Fortran source code.
Because it is an extension to the LLVM compiler, it's advantages are many
fold. Namely, it is very fast, it completely abstracts the complexity of
numerical analysis from the target code of the instrumentation, and is fairly
easy to plug into a user's existing program. Given that NumPy, PyTorch and
FrugallyDeep are all implemented in C++, Verificarlo allowed us to
instrument those frameworks and run predictions using MCA.

\subsection{NumPy}
To evaluate the stability of networks using NumPy, we reimplemented a neural
network data structure that offers an interface for flexible network
architecture descriptions (similar to TensorFlow and PyTorch). For now, our
implementation offers Dense and Convolutional layers only but it can certainly
be extended to offer a wider range of layers. 
\paragraph{}
For MNIST, we trained a network containing a single hidden fully-connected layer
which achieved 97.74\% accuracy. Then, we ran $n$ MCA trials on a subset of the
test set, because doing so on the totality of the set (10000 images) was not
feasible given our resources. As per [refer to verificarlo paper] one can
approximate the number of significant digits in base 10 of a computation with
the following formula $-{log_{10}(\frac{\sigma}{|\mu|})}$, where $\sigma$ is the
standard deviation and $\mu$ is the mean of the distribution over $n$ trials.
Furthermore, Verificarlo allows one to run MCA trials with specified precision,
i.e the number of bits retained in the mantissa. With full precision, 53 bit
mantissa for IEEE 754-2008 (binary64), running 20 MCA trials on the first 100
test images resulted in an average of 14 significant figures base 10 whereas 45
bit precision gives on average 11 sigfigs$_{10}$.
\paragraph{}
For CIFAR10, we trained a simple CNN composed solely of three convolutional
layers and one flatten layer. This simple architecture could not come close to
state-of-the-art accuracies (50-60\% vs. 97\%). However, by performing 40 trials
with binary64 precision of 53 we found that the predicted outputs averaged 13
sigfigs$_{10}$. With a reduced precision of 45 bits instead of 53, the network
still outputed an average of 11 significant figures base 10.

\subsection{FrugallyDeep}
In application, scientists and engineers alike rarely use a custom
implementation of neural networks to solve their problems. Instead they rely on
robust frameworks like TensorFlow and PyTorch. However, even if these frameworks
are built in C++ instrumenting them with Verificarlo is no trivial task. So it
is difficult to test the stability of networks implemented in these frameworks.
This is where FrugallyDeep comes in. It is a very lightweight headers-only C++
library designed to allow users to run predictions on TensorFlow models without
needing to have TensorFlow as their project's dependencies. Frugally translates
the .h5 model format into a custom JSON structure that can then perform those
predictions. This library can be trivially instrumented with Verificarlo. So
far, we can run MCA on TensorFlow models but our results aren't to be trusted
yet. Without further investigation, we can't be sure that the stability or
lackthereof is a cause of FrugallyDeep's implementation or whether it is
intrinsic to the network's themselves.
\subsection{PyTorch}

\section{Results and Future work}
So far, our results suggest that the networks implemented in this study seem to
be stable. This might be due to the simplicity of the use cases studied and we
can't assume that the same would hold for different use cases or achitectures.
It would be useful to replicate the PyTorch work for TensorFlow to allow for the
testing of many more use cases without needing to reinvent the wheel in our own
NumPy implementation. Both in NumPy and FrugallyDeep, training and inference was
done on CPUs. It would be useful to see how those results change if we use GPUs
instead.
\end{document}